\documentclass{article}
\usepackage{graphicx}
\usepackage{float}

\graphicspath{ {images/} }

\newcommand{\projectnaam}{Software Reengineering}
\newcommand{\student}{Van Muylder Ben \& Geeraert Lander}
\title{\textmd{\textbf{Final Project Report}}\\\normalsize\vspace{0.1in}\Large{\projectnaam}}
\author{\student}\date{\today}

\setcounter{section}{-1}

\begin{document}
\maketitle
\newpage

\section{Introduction}

The purpose of this project is to plan a refactoring for an existing software project such that we can easily add newly requested features.
The exiting software for this project is JFreeChart, and the functionality which we wish to add is the possibility to (1) have a plot where each data point is a different shape and (2) have the ability to read all different shapes from a database.\\

To be clear, we are not meant to implement these features, rather we are to refactor the project such that it would be easy for a developper to add these features.

\section{Plan}

We will start by informing ourselves of the current structure and workings of this library. More specifically, we will figure out where and how exactly plots are generated in the code.\\

Using CodeScene, we looked at which classes it suggested needed refactoring (classes which were hotspots). This lead us to the XYPlot Class, which effectively is the class which renders plots.\\

XYPlot then uses a XYItemRenderer, or rather a class which implements the XYItemRenderer interface to effectively render its plots. Such a class could be XYLineAndShapeRenderer, which effectively renders a plot with shapes and lines which connects these shapes.\\

However, a XYLineRenderer is not capable of rendering a data set with all different shapes (which is were the assignment comes in place).\\

As such we can note that the ability for a serie of a plot to have a shape is already present it the code. This specific part is contained in the AbstractRenderer class. However, the class only allows you to set one specific shape for a serie. This is where we presume that our initial refactoring will take place.\\

For now we plan to split up the shape setting code/function to allow more flexible shape setting, for example: using multiple functions such as SetShapeUnique, SetShapeByType (where the type could be random)...


\section{Current Implementation}


\newpage
\section{Conclusion}


\end{document}